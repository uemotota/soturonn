Skip to content
Search or jump to…

Pull requests
Issues
Marketplace
Explore
 
@uemotota 
Learn Git and GitHub without any code!
Using the Hello World guide, you’ll start a branch, write comments, and open a pull request.


0
00ren51/TEX
 Code Issues 0 Pull requests 0 Actions Projects 0 Wiki Security Insights
TEX/zemi6.tex
@ren51 ren51 Add files via upload
8f6fe27 2 minutes ago
91 lines (85 sloc)  5.84 KB
  
\documentclass{jsarticle}
\makeindex
\usepackage[dvipdfmx]{graphicx}
\usepackage{dcolumn}
\usepackage{graphicx}


\begin{document}
\rightline{2019/05/28}
\vspace{5em}
\fontsize{24pt}{40pt}\selectfont
\centerline{調べた文献6}
\vspace{1em}
\fontsize{10.5pt}{20pt}\selectfont
\section{選んだ理由}
コミュニケーションというキーワードと自分もアレルギー性鼻炎だったため.

\section{タイトル}
種類:論文
\\
「アレルギー性鼻炎理解のための罹患者と健常者のコミュニケーションの在り方」
\section{研究の背景}
全国の耳鼻咽喉科医及びその家族を対象とした全国疫学調査によると,アレルギー性鼻炎罹患者は1998年の調査では29.8%であったのに対し,2008年では39.8%となり10年間で10%増加している.アレルギー性鼻炎は他のアレルギー疾患と比較すると患者数が多いが,生命に関わる病気でないことから軽視されていまっている.
\section{問題点}
その症状は身体的精神的機能へ影響を及ぼしQOL(Quality Of Life)を低下させるものであり,適切な支援を必要とする.アレルゲンの除去は罹患者にとって症状を緩和させるために重要なのもであり,周囲の協力や理解が必要である.
\section{目的}
本研究ではアレルギー性鼻炎に対する知識や対処について調査し,罹患者と健常者のアレルギー性鼻炎に対する意識を明らかにする.
\section{研究で行ったこと}
A大学に通う学生328名を対象に質問紙調査を実施した.調査期間は2017年10月下旬から11月中旬であり,有効回答は323名であった.
対象者の内訳は以下に示す.
\\
\begin{table}[h]
\centering
\caption{性別}
\begin{tabular}{lrr}\hline
~~~~~~~ & n & \% \\ \hline\hline
男性~~~~~~~ & 134 & 41.5 \\ \hline
女性~~~~~~~ & 189 & 58.5 \\ \hline
\end{tabular}
\end{table}

\section{結果}
\subsection{対象者の罹患率}
対象者のアレルギー性鼻炎であると答えたのは全体の54.8\%であり,いいえが45.2\%だった.
\subsection{発症年齢と発現時期}
発症年齢として最も多かったのが10歳未満で45.2\%であり,発症時期として最も多かった回答が3〜5月で55.4\%だった.
\subsection{症状について}
アレルギー性鼻炎の症状として複数回答可で答えてもらったところ,「鼻水」,「鼻詰まり」,「くしゃみ」が最も多く,その他としては頭痛,だるさ,涙などが挙げられていた.
\subsection{罹患率と健常者のアレルギー性鼻炎に対する印象}
アレルギー性鼻炎に対する知識や理解について明らかにすることを目的し,対象者全員に「アレルギー性鼻炎に対してどのような印象を持っていますか」という質問をした.その結果「鼻水がつらい」,「鼻づまりがつらい」,「くしゃみがつらい」の3項目で「当てはまる」と答えたものが多かった.また健常者の7割以上も「当てはまる」と回答しておりアレルギー性鼻炎のつらさは理解していることがわかった.
\begin{table}[h]
\centering
\caption{アレルギー性鼻炎に対する印象}
\scalebox{0.7}[1.0]{
\begin{tabular}{lcccccc}\hline
 & &よく当てはまる(\%)&やや当てはまる&どちらともいえない&あまりあ当てはまらない&ほとんど当てはまらない\% \\

 \hline\hline
鼻水がつらい~~** & 罹患者 & 78.0 & 18.1 & 1.1 & 2.8 & 0.0\\ 
                 & 健常者 & 74.0 & 13.0 & 2.1 & 2.1 & 8.9  \\ \hline
鼻づまりがつらい~~* & 罹患者 & 73.4 & 19.2 & 2.8 & 3.4 & 1.1\\ 
                 & 健常者 & 63.0 & 21.9 & 3.4 & 4.1 & 7.5  \\ \hline
くしゃみがつらい~~* & 罹患者 & 60.5 & 24.9 & 4.0 & 8.5 & 2.3\\ 
                 & 健常者 & 52.1 & 22.6 & 9.6 & 6.2 & 9.6  \\ \hline
匂いがわからない~~ & 罹患者 & 37.4 & 30.5 & 10.7 & 10.2 & 10.7\\ 
                 & 健常者 & 27.4 & 28.8 & 14.4 & 11.0 & 18.5  \\ \hline 
味がよくわからなくなる~~ & 罹患者 & 24.9 & 25.4 & 17.5 & 12.4 & 19.8\\ 
                 & 健常者 & 19.9 & 30.1 & 10.3 & 19.9 & 19.9  \\ \hline
集中力が低下する~~ & 罹患者 & 40.1 & 31.1 & 14.1 & 6.2 & 8.5\\ 
                 & 健常者 & 30.8 & 32.9 & 13.7 & 8.9 & 13.7  \\ \hline
疲れやすくなる~~ & 罹患者 & 27.7 & 31.1 & 19.2 & 12.4 & 9.6\\ 
                 & 健常者 & 21.9 & 30.8 & 18.5 & 14.4 & 14.4  \\ \hline
よく眠れない~~ & 罹患者 & 35.0 & 28.8 & 14.7 & 10.7 & 10.7\\ 
                 & 健常者 & 35.6 & 29.5 & 8.2 & 14.4 & 12.3  \\ \hline
のどに違和感がある~~*** & 罹患者 & 26.0 & 24.9 & 19.8 & 15.3 & 14.1\\ 
                 & 健常者 & 10.3 & 20.5 & 18.5 & 32.2 & 18.5  \\ \hline
日中に眠気を感じる~~*** & 罹患者 & 33.9 & 23.2 & 20.3 & 13.0 & 9.6\\ 
                 & 健常者 & 11.0 & 20.5 & 25.3 & 20.5 & 22.6  \\ \hline 
のどが渇きやすい~~*** & 罹患者 & 21.5 & 22.0 & 24.3 & 19.3 & 13.0\\ 
                 & 健常者 & 6.2 & 14.4 & 22.6 & 29.5 & 27.4  \\ \hline
\end{tabular}
}
\end{table}
\section{結論}
アレルギー性鼻炎の知識や理解について,罹患者が増加傾向にあり情報に接する機会も増え症状についての理解が進んでいるが,のどの症状や薬の副作用についての健常者の理解は進んではいなかった。
\section{感想}
今回はコミュニケーションとは関係がなかったが表があることで理解はしやすかった.
\section{参考文献} 
石原研治,海老沢幸恵:”アレルギー性鼻炎理解のための罹患者と健常者のコミュニケーションの在り方”,茨城大学教育学部紀要,68号,379-386ページ,2019
\end{document}

\UTF{00A9} 2020 GitHub, Inc.
Terms
Privacy
Security
Status
Help
Contact GitHub
Pricing
API
Training
Blog
About
